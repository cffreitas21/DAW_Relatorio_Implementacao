\chapter{Conclusão e Trabalho Futuro}
\label{cap4}
O desenvolvimento da fase de análise e desenho do projeto Streambolt TV permitiu estabelecer uma base sólida para a implementação de uma plataforma digital dedicada à organização e consulta de filmes. Foram definidos os requisitos essenciais do sistema, detalhados os principais casos de uso e desenvolvidos modelos de dados e protótipos de interface que respondem às necessidades dos utilizadores identificados. Este processo contribuiu para clarificar as funcionalidades essenciais da aplicação e assegurar a sua viabilidade técnica, proporcionando uma visão global e coerente do produto a desenvolver.

Na fase seguinte do projeto, pretende-se avançar para a implementação prática do sistema Streambolt TV, consolidando os resultados obtidos na fase de análise e desenho. Esta etapa incluirá o desenvolvimento de uma API para gerir os dados e regras de negócio, fornecendo suporte aos casos de uso definidos. Em simultâneo, será criada uma aplicação segundo a arquitetura Model-View-Controller (MVC), que integrará as vistas do frontend com os dados da API. As vistas poderão ser geradas pelo motor de templates da tecnologia escolhida (Laravel/Blade, ASP.NET/Razor) ou por uma Single Page Application (SPA) em React ou Angular. A API será responsável pela conexão com a base de dados e poderá incluir mecanismos de autenticação e análise de uso, garantindo uma interface funcional e intuitiva para os utilizadores.