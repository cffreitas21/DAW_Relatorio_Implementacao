\chapter{Introdução}
\label{intro}
O projeto Streambolt TV visa o desenvolvimento de uma plataforma web que funciona como uma biblioteca digital de filmes, permitindo a organização, pesquisa e consulta detalhada de uma coleção de títulos audiovisuais. Este sistema pretende oferecer aos utilizadores uma interface intuitiva e funcional para aceder às informações sobre os filmes, como títulos, sinopses, gêneros e anos de lançamento, sem funcionalidades de reprodução incorporadas.

Esta fase inicial de análise e desenho constitui um passo essencial para   compreender as necessidades do sistema e estruturar a solução de forma     coerente e organizada. Durante esta etapa, foram identificados os atores e os casos de uso que descrevem o comportamento esperado, elaborados protótipos das interfaces e desenvolvido o modelo de dados que suportará a aplicação. Estas atividades permitem clarificar os requisitos funcionais e estabelecer uma base sólida e estruturada para o desenvolvimento futuro.

O relatório está organizado em três capítulos principais: o segundo capítulo aborda a análise do sistema, incluindo os requisitos e a definição dos casos de uso; o terceiro capítulo apresenta o desenho do sistema, destacando os diagramas UML, o modelo de dados e os protótipos das interfaces; finalmente, o quarto capítulo contempla as conclusões e as perspetivas para as fases subsequentes do projeto.

