\chapter{Implementação}
\label{cap5}

\clearpage
\section{Arquitetura do Sistema}
Na seguinte figura podemos observar o diagrama de implantação do 
sistema StreamBoltTV onde se demonstram claramente os blocos de SW e HW que compõem o sistema.

\begin{figure}[H]
    \centering
    \includegraphics[width=0.8\textwidth]{figuras/Base de Dados/Diagrama_Implantacao.png}
    \caption{Diagrama de Implantação}
    \label{fig:DI}
\end{figure}

\section{Tecnologias Usadas}

O projeto StreamBoltTV utiliza diversas tecnologias de suporte ao desenvolvimento,
 incluindo linguagens de programação, frameworks, bases de dados e ferramentas de build.

\subsection{Linguagens de Programação}

\begin{itemize}
    \item \textbf{PHP 8.2.12}: Lógica de negócio, processamento de dados e interação com base de dados. 
    
    \url{https://www.php.net/}
    
    \item \textbf{JavaScript (ES6+)}: Interatividade no frontend e requisições assíncronas. 
    
    \url{https://developer.mozilla.org/en-US/docs/Web/JavaScript}
\end{itemize}

\subsection{Frameworks Principais}

\begin{itemize}
    
    \item \textbf{Laravel 12.45.2}: Estrutura MVC da aplicação, gestão de rotas, autenticação e ORM. 
    
    \url{https://laravel.com/docs/12.x}
    
    \item \textbf{Tailwind CSS 3.1.0}: Estilos e design responsivo do frontend. 
    
    \url{https://tailwindcss.com/docs}
    
    \item \textbf{Alpine.js 3.4.2}: Comportamento reativo leve no frontend.
    
    \url{https://alpinejs.dev/start-here}
\end{itemize}

\subsection{Base de Dados}

\begin{itemize}
    \item \textbf{MySQL / PostgreSQL}: Armazenamento persistente de dados. 
    
    \url{https://www.mysql.com/}, \url{https://www.postgresql.org/}
    
    \item \textbf{Eloquent ORM}: Abstração da base de dados e mapeamento objeto-relacional. 
    
    \url{https://laravel.com/docs/12.x/eloquent}
\end{itemize}

\subsection{Ferramentas de Build e Desenvolvimento}

\begin{itemize}
    
    \item \textbf{Vite 7.0.7}: Compilação de assets e otimização do frontend.
    
    \url{https://vitejs.dev/guide/}
    
    \item \textbf{Composer}: Gestão de dependências PHP. 
    
    \url{https://getcomposer.org/doc/}
    
    \item \textbf{NPM}: Gestão de dependências JavaScript. 
    
    \url{https://docs.npmjs.com/}
\end{itemize}

\subsection{Servidor Web e Infraestrutura}

\begin{itemize}
    \item \textbf{Apache / Nginx}: Servidor web para processamento de requisições HTTP/HTTPS.
    
    \url{https://httpd.apache.org/}, \url{https://nginx.org/}
\end{itemize}

\subsection{Autenticação e Segurança}

\begin{itemize}
    \item \textbf{Laravel Sanctum 4.0}: Autenticação de API e SPA. 
    
    \url{https://laravel.com/docs/12.x/sanctum}
    
    \item \textbf{Laravel Breeze 2.3}: Login, registo e gestão de passwords. 
    
    \url{https://laravel.com/docs/12.x/starter-kits#laravel-breeze}
\end{itemize}

\subsection{Comunicação HTTP}

\begin{itemize}
    \item \textbf{Axios 1.11.0}: Requisições AJAX para API REST. 
    
    \url{https://axios-http.com/docs/intro}
\end{itemize}

\clearpage
\subsection{Padrões Arquiteturais}

\begin{itemize}
    \item \textbf{MVC}: Organização do código em Models, Views e Controllers. 
    
    \url{https://laravel.com/docs/12.x/structure}
    
    \item \textbf{RESTful API}: Endpoints CRUD. 
    
    \url{https://laravel.com/docs/12.x/routing#api-routes}
    
    \item \textbf{Eloquent ORM}: Models representam tabelas e relações. 
    
    \url{https://laravel.com/docs/12.x/eloquent}
\end{itemize}

\clearpage
\section{Desenvolvimento da API}

\subsection{Especificação da interface}

A tabela a seguir apresenta os endpoints da API REST do StreamBoltTV.

\begin{figure}[H]
    \centering
    \includegraphics[width=0.9\textwidth]{figuras/Endpoints.png}
    \caption{Endpoints da API}
    \label{fig:ERM}
\end{figure}

\clearpage
\noindent\textbf{Notas importantes:}
\begin{itemize}
    \item Todos os comentários criados começam como não aprovados.
    \item Apenas administradores podem aprovar ou rejeitar comentários.
    \item Ao apagar filme, os comentários associados são apagados automaticamente.
    \item Ao apagar filme, o poster é removido do storage.
\end{itemize}

\subsection{Decisões de Implementação da API}

\subsubsection{Abordagem Code-First com Migrations}

O projeto adota a abordagem \textbf{code-first} usando migrations do Laravel, em vez de criar a base de dados manualmente (database-first).  

Esta abordagem tem várias vantagens:

\begin{itemize}
    \item \textbf{Versionamento integrado:} A estrutura da base de dados evolui junto com o código, permitindo tracking e rollback de alterações.
    \item \textbf{Portabilidade:} O schema é independente do SGBD, facilitando migrações entre MySQL, PostgreSQL ou outros suportados pelo Laravel.
    \item \textbf{Reversibilidade:} Cada migration possui métodos para criar (\texttt{up()}) e desfazer (\texttt{down()}) alterações.
    \item \textbf{Desenvolvimento iterativo:} Facilita alterações incrementais, como adicionar campos ou modificar relações, sem comprometer dados existentes.
\end{itemize}

\clearpage
\subsubsection{Arquitetura RESTful}

A API segue rigorosamente padrões REST:

\begin{itemize}
    \item Recursos representados por URLs no plural (\texttt{/movies}, \texttt{/comments}).
    \item Métodos HTTP semânticos (GET, POST, DELETE) e respostas em JSON.
    \item Separação clara entre rotas da API (\texttt{routes/api.php}) e rotas web (\texttt{routes/web.php}).
\end{itemize}

\subsubsection{Design de Controladores}

Controladores organizados por namespace (\texttt{App/Http/Controllers/Api}) e seguindo métodos RESTful padrão:

\texttt{index()}, \texttt{show()}, \texttt{store()}, \texttt{update()}, \texttt{destroy()}, entre outros.

\subsubsection{Segurança do Sistema}

\begin{itemize}
    \item Autenticação via Laravel Sanctum (tokens para SPAs).
    \item Autorização via middleware de roles (admin / streamer).
\end{itemize}

\clearpage
\subsection{Principais casos relevantes de programação}

O desenvolvimento do sistema envolveu casos de programação que combinam 
segurança, lógica de negócio e interatividade no frontend. 
Nesta secção são apresentados alguns casos acompanhados de pequenos trechos de código.

\paragraph{1. Autenticação e controlo de acesso por role:}

Para garantir que apenas utilizadores autorizados executam operações críticas
 (criação e remoção de filmes, aprovação de comentários), foi implementado 
 um sistema de controlo de acesso baseado em roles (administrador ou cinéfilo), 
 utilizando middleware.


\begin{itemize}
    \item \textbf{Middleware de verificação de administrador:}
\begin{lstlisting}[language=php]
if (!auth()->check() || !auth()->user()->isAdmin()) {
    abort(403, 'Access denied.');
}
\end{lstlisting}

    \item \textbf{Métodos helper no model User:}
\begin{lstlisting}[language=php]
public function isAdmin(): bool {
    return $this->role === 'admin';
}

public function isStreamer(): bool {
    return $this->role === 'streamer';
}
\end{lstlisting}

    \item \textbf{Aplicação do middleware nas rotas:}
\begin{lstlisting}[language=php]
Route::middleware(['auth','admin'])->group(function () {
    Route::post('/movies', [MovieController::class, 'store']);
    Route::delete('/movies/{id}', [MovieController::class, 'destroy']);
});
\end{lstlisting}
\end{itemize}

\noindent Esta abordagem garante segurança em camadas, reutilização de código e uma separação clara de responsabilidades.

\paragraph{2. Filtragem dinâmica de filmes na homepage do streamer:}

A homepage do streamer permite filtrar filmes por género e ano sem recarregar a página, 
utilizando JavaScript e a API REST.

\begin{itemize}
    \item \textbf{Carregamento inicial dos filmes via API:}
\begin{lstlisting}[language=Java]
allMovies = await fetch('/api/movies')
    .then(response => response.json());
renderMovies(allMovies);
\end{lstlisting}

    \item \textbf{Aplicação de filtros cumulativos:}
\begin{lstlisting}[language=Java]
filteredMovies = allMovies
    .filter(m => !genre || m.genre === genre)
    .filter(m => !year || 
        new Date(m.release_date).getFullYear() == year);

renderMovies(filteredMovies);
\end{lstlisting}

    \item \textbf{Reposição dos filtros:}
\begin{lstlisting}[language=Java]
genreFilter.value = '';
yearFilter.value = '';
renderMovies(allMovies);
\end{lstlisting}
\end{itemize}

\noindent Esta solução melhora significativamente a performance e a experiência do utilizador, evitando múltiplas requisições ao servidor e mantendo a lógica de filtragem simples e eficiente.


