\chapter{Implementação}
\label{cap5}

\clearpage
\section{Arquitetura do Sistema}
Na seguinte figura podemos observar o diagrama de implantação do 
sistema StreamBoltTV onde se demonstram claramente os blocos de SW e HW que compõem o sistema.

\begin{figure}[H]
    \centering
    \includegraphics[width=0.8\textwidth]{figuras/Base de Dados/Diagrama_Implantacao.png}
    \caption{Diagrama de Implantação}
    \label{fig:DI}
\end{figure}

\section{Tecnologias Usadas}

O projeto StreamBoltTV utiliza diversas tecnologias de suporte ao desenvolvimento, incluindo linguagens de programação, frameworks, bases de dados e ferramentas de build.

\subsection{Linguagens de Programação}

\begin{itemize}
    \item \textbf{PHP 8.2.12}: Lógica de negócio, processamento de dados e interação com base de dados. 
    
    \url{https://www.php.net/}
    
    \item \textbf{JavaScript (ES6+)}: Interatividade no frontend e requisições assíncronas. 
    
    \url{https://developer.mozilla.org/en-US/docs/Web/JavaScript}
\end{itemize}

\subsection{Frameworks Principais}

\begin{itemize}
    
    \item \textbf{Laravel 12.45.2}: Estrutura MVC da aplicação, gestão de rotas, autenticação e ORM. 
    
    \url{https://laravel.com/docs/12.x}
    
    \item \textbf{Tailwind CSS 3.1.0}: Estilização e design responsivo do frontend. 
    
    \url{https://tailwindcss.com/docs}
    
    \item \textbf{Alpine.js 3.4.2}: Comportamento reativo leve no frontend.
    
    \url{https://alpinejs.dev/start-here}
\end{itemize}

\subsection{Base de Dados}

\begin{itemize}
    \item \textbf{MySQL / PostgreSQL}: Armazenamento persistente de dados. 
    
    \url{https://www.mysql.com/}, \url{https://www.postgresql.org/}
    
    \item \textbf{Eloquent ORM}: Abstração da base de dados e mapeamento objeto-relacional. 
    
    \url{https://laravel.com/docs/12.x/eloquent}
\end{itemize}

\subsection{Ferramentas de Build e Desenvolvimento}

\begin{itemize}
    
    \item \textbf{Vite 7.0.7}: Compilação de assets e otimização do frontend.
    
    \url{https://vitejs.dev/guide/}
    
    \item \textbf{Composer}: Gestão de dependências PHP. 
    
    \url{https://getcomposer.org/doc/}
    
    \item \textbf{NPM}: Gestão de dependências JavaScript. 
    
    \url{https://docs.npmjs.com/}
\end{itemize}

\subsection{Servidor Web e Infraestrutura}

\begin{itemize}
    \item \textbf{Apache / Nginx}: Servidor web para processamento de requisições HTTP/HTTPS.
    
    \url{https://httpd.apache.org/}, \url{https://nginx.org/}
\end{itemize}

\subsection{Autenticação e Segurança}

\begin{itemize}
    \item \textbf{Laravel Sanctum 4.0}: Autenticação de API e SPA. 
    
    \url{https://laravel.com/docs/12.x/sanctum}
    
    \item \textbf{Laravel Breeze 2.3}: Login, registo e gestão de passwords. 
    
    \url{https://laravel.com/docs/12.x/starter-kits#laravel-breeze}
\end{itemize}

\subsection{Comunicação HTTP}

\begin{itemize}
    \item \textbf{Axios 1.11.0}: Requisições AJAX para API REST. 
    
    \url{https://axios-http.com/docs/intro}
\end{itemize}

\clearpage
\subsection{Padrões Arquiteturais}

\begin{itemize}
    \item \textbf{MVC}: Organização do código em Models, Views e Controllers. 
    
    \url{https://laravel.com/docs/12.x/structure}
    
    \item \textbf{RESTful API}: Endpoints CRUD. 
    
    \url{https://laravel.com/docs/12.x/routing#api-routes}
    
    \item \textbf{Eloquent ORM}: Models representam tabelas e relações. 
    
    \url{https://laravel.com/docs/12.x/eloquent}
\end{itemize}

\clearpage
\section{Desenvolvimento da API}

\subsection{Especificação da interface}

A tabela a seguir apresenta os endpoints da API REST do StreamBoltTV.

\begin{figure}[H]
    \centering
    \includegraphics[width=0.9\textwidth]{figuras/Endpoints.png}
    \caption{Endpoints da API}
    \label{fig:ERM}
\end{figure}

\clearpage
\noindent\textbf{Notas importantes:}
\begin{itemize}
    \item Todos os comentários criados começam como não aprovados.
    \item Apenas administradores podem aprovar ou rejeitar comentários.
    \item Ao apagar filme, os comentários associados são apagados automaticamente.
    \item Ao apagar filme, o poster é removido do storage.
\end{itemize}
